\chapter*{\textit{GLOSARIO}}
Se se presentan algunos términos utilizados en el trabajo desarrollado:
\begin{description}
\item[CR:] Radiación cósmica, son partículas subatómicas de origen extraterrestre que impactan nuestro planeta. Este flujo está compuesto por aproximadamente 90\% de protones y 9\% de partículas alfa, el resto son electrones e iones pesados \footcite{procureur2018muon}.

\item[EAS:] Por sus siglas en inglés Extensive Air Shower, se le llama a la cascada de millones de partículas generadas cuando un rayo cósmico interactúa con los átomos terrestres.

\item[Muón:] El muón es una partícula elemental, cuya masa es $\sim$ 200 veces la del electrón: 105,6 MeV/$c^2$. 
\item[ToF: ] Tiempo de vuelo.
\item[MuTe:] Telescopio de Muones, proyecto el cual lleva a cabo un estudio muográfico de volcanes en Colombia \footnote{ \url{http://halley.uis.edu.co/fuego/}}.

\item[WCD:] Detectores Cherenkov de agua. Es un detector de partículas cargadas conformado por un volumen de agua ultra pura y un elemento sensible (foto-multiplicador)
\item[GMM: ] Modelos probabilísticos que representan subpoblaciones normalmente distribuidas dentro de una población general.
\item[PDF: ] Función de densidad de probabilidad.

\end{description}
