% ------------------------------------------------------------------------
% ------------------------------------------------------------------------
% ------------------------------------------------------------------------
%                                Abstract
% ------------------------------------------------------------------------
% ------------------------------------------------------------------------
% ------------------------------------------------------------------------
\chapter*{ABSTRACT}

\footnotesize{
\begin{description}
  \item[TITLE:] DISCRIMINATION OF BACKGROUND NOISE IN MUOGRAPHY USING MACHINE LEARNING TECHNIQUES \astfootnote{Bachelor Thesis}
  \item[AUTHOR:] ALEJANDRO RAMIREZ MUÑOZ, DAVID VILLABONA ARDILA \asttfootnote {Faculty of Physical-Mechanical Engineering. School of Systems Engineering and Informatics. Director: Jesús Peña Rodríguez}
  \item[KEYWORDS:] Cosmic rays,  Muography, machine learning, background noise, , MuTe.
  \item[DESCRIPTION:]\hfill 
  
Muongraphy is a non-invasive technique used to scan large anthropic or natural structures. Its operating principle consists of measuring the flux of muons that cross the structure in different directions. This technique has applications in fields such as: underground measurements \footcite {Tanaka2005, Tanaka2009, Lesparre2010, Lesparre2011, Lesparre2012}, archeology \footcite {Morishima2017, Gmez2016, Alvarez1970}, detection of hidden materials in containers, reactors and nuclear waste \footcite { Fujii2013}. 

This technique is affected by an underestimation of the object density, as a consequence of background noise (false-positives) that can be classified into: charged particles from extensive air showers (EAS) \footcite {Nishiyama2014, Gomez2017}, the incident particles from the rear of the detector, low energy mouns that are scattered by the surface of the volcano and multiple particle events \footcite {nishiyama2014experimental}. 

Techniques have been developed for the elimination of noise, such as the installation of absorbent panels, to filter low-energy particles and increasing the number of sensitive panels, to reduce the probability of detecting combinational events  \footcite {Lesparre2012}. At present, the elimination of background noise with ToF systems and particle identification  is being considered \footcite {mounes that impact the detector from the rear.} \Footcite {Marteau2014, Cimmino2017}.




\end{description}}\normalsize
% ------------------------------------------------------------------------ 