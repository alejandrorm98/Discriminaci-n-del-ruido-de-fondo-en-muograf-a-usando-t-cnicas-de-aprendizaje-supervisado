\chapter{\MakeUppercase{Metodología}}
\label{ch:e_lento}


\subsection{Extracción de características. \\ \\}

Se busca realizar una parametrización y análisis de los datos suministrados por el WCD por medio de un histograma de carga. A continuación se presentan los pasos empleados:

\begin{itemize}
    \item Extraer los datos crudos del detector y posteriormente ordenarlos.
     \item Parametrizar los datos para ajustarlo a una distribución probabilística.
     \item Implementar un histograma de carga, para identificar y separar la componente muónica de la electromagnética y multipartícula.
     \item Encontrar características según los resultados de las variables de entrada de cada componente.
 \end{itemize}
 

\subsection{Ajuste de características para desarrollar clasificadores supervisados que discriminan entre mounes, EM y multipartícula. \\ \\}

Se va a identificar las características para ajustarlas a diferentes modelos de clasificación y etiquetarlas entre las componentes estudiadas en el proyecto. A continuación se presentan las actividades correspondientes:
\begin{itemize}
    \item Definir una función de probabilidad optimizando los valores que ajusten a la curva de la distribución seleccionada.
    \item A partir de los valores ajustados con la función de densidad de probabilidad (PDF), analizar independientemente cada componente por medio de la distribución extraída del histograma de carga.
    \item Evaluando las componentes independientes, se etiqueta cada tupla de datos con unos y ceros, mounes y EP respectivamente.
    \item Para el entrenamiento de los datos con el modelo a emplear, se usan características y etiquetas, de tal manera que se pueda evaluar el comportamiento del clasificador a usar\footcite{perez2005modelos}.
\end{itemize}

\subsection{Implementación del clasificador no supervisado para discriminar los mounes de bajo momentum.\\ \\}

Se comprueban algoritmos no supervisados de clasificación usando previamente la discriminación de la componente electromagnética de la muónica, para separar ahora entre muón de alto y de bajo momentum ($<$ 1 GeV/c). Este proceso se llevará a cabo después de comparar sus tiempos de vuelo como característica.  \\



\subsection{Comparación y validación de todos los datos obtenidos a lo largo del proyecto de investigación.\\ \\ }

Esta fase final divide en:

\begin{itemize}
    \item Comparación de los clasificadores supervisados por métricas de precisión.
    \item Analizar los datos obtenidos por los clasificadores no-supervisados con métricas de precisión.
    \item Encontrar y diferenciar el mejor resultado de cada clasificador, optimizando el resultado.
\end{itemize}

