% ------------------------------------------------------------------------
% ------------------------------------------------------------------------
% ------------------------------------------------------------------------
%                                Resumen
% ------------------------------------------------------------------------
% ------------------------------------------------------------------------
% ------------------------------------------------------------------------
\chapter*{RESUMEN}

\footnotesize{
\begin{description}
  \item[TÍTULO:] DISCRIMINACIÓN DEL RUIDO DE FONDO EN MUOGRAFÍA USANDO TÉCNICAS DE APRENDIZAJE AUTOMATIZADO. \astfootnote{Trabajo de grado}
  \item[AUTOR:] ALEJANDRO RAMIREZ MUÑOZ, DAVID VILLABONA ARDILA \asttfootnote {Faculty of Physical-Mechanical Engineering. School of Systems Engineering and Informatics. Director: Jesús Peña Rodríguez}
  \item[PALABRAS CLAVE:] Rayos cósmicos, Muografía, Aprendizaje automatizado, ruido de fondo.
  \item[DESCRIPCIÓN:]\hfill 
  
  La muografía es una técnica no-invasiva que se utiliza para escanear grandes estructuras antrópicas o naturales. Su principio de funcionamiento consiste en la medición del flujo de muones que cruzan la estructura en diferentes direcciones. Esta técnica tiene aplicaciones en campos tales como: mediciones subterráneas\footcite{Tanaka2005, Tanaka2009, Lesparre2010, Lesparre2011, Lesparre2012}, arqueología\footcite{Morishima2017, Gmez2016, Alvarez1970}, detección de materiales ocultos en contenedores, reactores  y residuos nucleares \footcite{Fujii2013}.\\

Esta técnica se ve afectada por una subestimación de la densidad del objeto, producto del ruido de fondo (falsos-positivos) que se pueden clasificar en: partículas cargadas procedentes de lluvias aéreas extensas (EAS) \footcite{Nishiyama2014,Gomez2017}, las partículas que inciden desde la parte trasera del detector, los muones de baja energía que son dispersados por la superficie del volcán y eventos de múltiple partícula\footcite{nishiyama2014experimental}.\\

Para la eliminación del ruido se han desarrollado técnicas pasivas como la instalación de paneles absorbentes, para filtrar las partículas de baja energía y el aumento de la cantidad de paneles sensibles, para disminuir la probabilidad de detectar eventos combinacionales \footcite{Lesparre2012}. En la actualidad se plantea la eliminación del ruido de fondo con  sistemas ToF e identificación de partículas \footcite{mounes que impactan en el detector por la parte de posterior.} \footcite{Marteau2014, Cimmino2017}. \\

En este trabajo se desarrolla un clasificador de aprendizaje automatizado que disminuya las principales fuentes de ruido que pueden afectar la muografía, basados en los datos del detector MuTe \footcite{inproceedings}. El proyecto se divide en 2 partes:

\begin{itemize}
    
    \item En la primera instancia se desarrolla un clasificador de aprendizaje supervisado para separar la componente electromagnética, muónica y de múltiple partícula.
    
    \item En la segunda parte se desarrolla un clasificador de aprendizaje no-supervisado el cual discrimina los muones de bajo momentum ($<$ 1 GeV/c).   
\end{itemize}
\end{description}}\normalsize
% ------------------------------------------------------------------------ 